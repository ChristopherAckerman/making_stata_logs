% Created 2020-07-15 Wed 13:04
% Intended LaTeX compiler: pdflatex
\documentclass[a4paper]{article}
\usepackage[utf8]{inputenc}
\usepackage[T1]{fontenc}
\usepackage{graphicx}
\usepackage{grffile}
\usepackage{longtable}
\usepackage{wrapfig}
\usepackage{rotating}
\usepackage[normalem]{ulem}
\usepackage{amsmath}
\usepackage{textcomp}
\usepackage{amssymb}
\usepackage{capt-of}
\usepackage{hyperref}
\usepackage{minted}
\usepackage[margin=0.5in]{geometry}
\date{\today}
\title{}
\hypersetup{
 pdfauthor={},
 pdftitle={},
 pdfkeywords={},
 pdfsubject={},
 pdfcreator={Emacs 28.0.50 (Org mode 9.3)}, 
 pdflang={English}}
\begin{document}

\tableofcontents

\begin{ABSTRACT}
This is a guide for formatting \texttt{Stata} output using emacs. The desired output is a PDF that clearly shows the commands run in \texttt{Stata}, along with the output from those commands. We'll be using \texttt{Emacs org-babel} to run code inside \texttt{.org} files. This makes it possible to re-run analyses, and also add nicely formatted prose to the same file. 
\end{ABSTRACT}

\section{Dependencies}
\label{sec:orgff04fd4}
\begin{center}
\begin{tabular}{lrl}
Program & My Version & Free\\
\hline
Stata & 15 & No\\
Emacs & 28.0.50 GNU Emacs & Yes\\
Latex & pdfTeX 3.14159265-2.6-1.40.21 & Yes\\
Git & 2.27.0 & Yes\\
Homebrew &  & Yes\\
latexmk & 4.69a & Yes\\
svn & 1.14.0 & Yes\\
\end{tabular}
\end{center}


This guide is specifically for \texttt{Stata .do} files---much of the information here will work for other languages that are supported by \texttt{org-babel} and \texttt{emacs-ESS}, but some of the export options may be different. You will need a working version of Stata with a license; I have Stata 15 and I assume that newer versions will be backwards-compatible, but I'm not sure about older versions. You will need a local copy of Stata; this approach will not work with a remote copy on a university's servers. 

You will also need a version of Emacs; I'm using the development version 28, but this should work for versions 24 or higher. You will also need a latex distribution to compile the final document. 

\section{Installing Emacs}
\label{sec:org3b61220}
This guide assumes you have never used \texttt{Emacs}, but have working installs of both \texttt{latex} and \texttt{Stata}. We will also be using \texttt{Homebrew}; if you don't have \texttt{Homebrew} installed and want to learn more, please visit \url{https://brew.sh/} and follow the installation instructions. I will also include installation instructions in this guide, so you can just run the command I list. The next step is to open a terminal. It's okay if you haven't used the terminal before; I'll list all the commands you'll need to run---you can just copy and paste them into the terminal. To launch the \texttt{Terminal} app on Mac, use the system tray to navigate to \texttt{Apps > Utilites > Terminal} and open \texttt{Terminal}.

\subsection{Install \texttt{Homebrew}}
\label{sec:orgdf12993}
If you don't have \texttt{Homebrew} installed, copy the following command into your terminal and run it.


\begin{verbatim}
/bin/bash -c "$(curl -fsSL https://raw.githubusercontent.com/Homebrew/install/master/install.sh)"
\end{verbatim}

\subsection{Install \texttt{git}}
\label{sec:org1e0491d}
   If you don't have \texttt{git} installed, run \texttt{brew install git} from your terminal.
If you already have \texttt{git} installed, you will get a message to that effect. You can also check directly by running
\texttt{which git} you get a filepath as output, that means that you already have \texttt{git} installed.

\subsection{Install \texttt{Emacs}}
\label{sec:orgcced5b6}

\begin{verbatim}
brew install emacs
\end{verbatim}

\subsection{Install \texttt{svn} (optional)}
\label{sec:orge5c27f4}
If you want to download my \texttt{emacs} configuration, you'll need \texttt{svn}.

\begin{verbatim}
brew install svn
\end{verbatim}

\section{Configuring \texttt{Emacs}}
\label{sec:org1a6f9b8}
I've built a minimal \texttt{emacs} configuration file that you can grab from github. You'll need to have \texttt{svn} installed to grab just the configuration subdirectory. First move to your home directory, then download the appropriate files.

\begin{verbatim}
cd 
svn checkout https://github.com/ChristopherAckerman/making_stata_logs/trunk/emacs/.emacs.d
\end{verbatim}
Most of the configuration is standard, but I've included a few custom \texttt{lisp} snippets to patch official packages. Removing these will break \texttt{Stata} support, but you can add additional packages without any issues (my actual \texttt{Emacs} configuration is much longer).
\end{document}
